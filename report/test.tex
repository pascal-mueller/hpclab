\documentclass[unicode,11pt,a4paper,oneside,numbers=endperiod,openany]{scrartcl}

\usepackage{amsmath}
\usepackage[table,xcdraw]{xcolor}
\usepackage{soul}
\sethlcolor{lightgray}
\usepackage{float}
\usepackage{tikz}
\usetikzlibrary{matrix}
\usepackage{amsfonts}

\input{assignment.sty}
\begin{document}


\setassignment
\setduedate{May 20, 2021, 12pm (midnight)}

\serieheader{High-Performance Computing Lab for CSE}
    {2021}
    {Student: Pascal Dominic Müller}
    {\\ \\ Discussed with: \\
        $\bullet$ Tarzis Maurer [tamaurer] \\
        $\bullet$ Zuzanna Herud [zherud] \\
    }
    {Solution for Project 6}
    {}
\newline

\assignmentpolicy

% -------------------------------------------------------------------------- %
% -------------------------------------------------------------------------- %
% --- Exercise 1 ----------------------------------------------------------- %
% -------------------------------------------------------------------------- %
% -------------------------------------------------------------------------- %

\section{Scientific Mathematical HPC Software Frameworks - The Poisson Equation [35 points]}

\subsection{Problem Statement: The Poisson Equation}

We solve the poisson equation on the 2 dimensional unit rectangle
$\Omega \subset \mathbb R^2=[0,1] \times [0,1]$ with vanishing dirichlet BC.
\begin{alignat}{3}
  -\Delta &y(x_1, x_2) &&= f(x_1, x_2) &&\text{  on  } \Omega = [0,1] \times [0,1]\\
          &y(x_1, x_2) &&= 0 &&\text{  on  } \partial \Omega
\end{alignat}

whereas $\Delta \equiv \frac{\partial^2}{\partial x_1^2} + \frac{\partial^2}{\partial x_2^2}$

\subsection{Discretization}


\subsubsection{Theory}
Let $g(x,y)$ be an at least twice continuous differentiable function. For the
discretization we use a central finite difference approach [1][Project 3].

The central finite difference in the direction of x (similar for y) is given by:

\begin{equation}
  \delta_h [g](x) = g(x+\frac{1}{2}h) - g(x-h)
\end{equation}

which leads to

\begin{equation}
  \frac{\partial g(x)^2}{\partial^2 x} = \dots = 
    \frac{g(x + h) - 2g(x) + g(x - h)}{h^2}
\end{equation}

\subsubsection{LHS}

Note: here $y$ denotes the seeked function, not the coordinate.

We first write out the laplacian operator.
\begin{equation}
  -\Delta y(x_1, x_2) = \frac{\partial y(x_1, x_2)^2}{\partial x_1^2} 
                      + \frac{\partial y(x_1, x_2)^2}{\partial x_2^2}
\end{equation}

Applying (4) to (5) for $x_1$ and $x_2$ separately, we get:
\begin{equation}
  \frac{\partial y(x_1, x_2)^2}{\partial x_1^2} = 
    \frac{y(x_1 + h, x_2) - 2y(x_1, x_2) + y(x_1 - h, x_2)}{h^2}
\end{equation}
\begin{equation}
  \frac{\partial y(x_1, x_2)^2}{\partial x_1^2} = 
    \frac{y(x_1, x_2 + h) - 2y(x_1, x_2) + y(x_1, x_2 - h)}{h^2}
\end{equation}

We plug (6) and (7) into (5) and use the notations (similar for $x_2$ and $j$
\begin{itemize}
  \item $y(x_1, x_2) = y_{i,j}$
  \item $y(x_1 + h, x_2) = y_{i+1, j}$
  \item $y(x_1 - h, x_2) = y_{i-h, j}$
\end{itemize}

to end up with

\begin{align}
  -\Delta y(x_1,x_2) & \\
                     &= \frac{y(x_1 + h, x_2) - 2y(x_1, x_2) + y(x_1 - h, x_2)}{h^2} \\
                     &\ \ + \frac{y(x_1, x_2 + h) - 2y(x_1, x_2) + y(x_1, x_2 - h)}{h^2} \\
                   &= \frac{1}{h^2} ( y_{i+1,j} - 2y_{i,j} + y_{i-1,j}
                      + y_{i, j+1} - 2y_{i,j} + y_{i,j-1}) \\
                   &= \frac{1}{h^2} ( -4_{i,j} + y_{i+1,j} + y_{i-1,j}
                      + _{i, j+1} + y_{i,j-1} )
\end{align}

We get:

\begin{equation}
  \boxed{
      -\Delta y(x_1,x_2) = \frac{1}{h^2} ( -4y_{i,j} + y_{i+1,j} + y_{i-1,j}
                           + y_{i, j+1} + y_{i,j-1} )
  }
\end{equation}

\subsubsection{RHS}
On the RHS we have a function $f(x_1, x_2): \mathbb R^2 \to \mathbb R$.

Its discretization is given by 

\begin{equation}
  \boxed{
    f_{i,j} := f(x_1^i, x_2^j)
  }
\end{equation}

\subsection{Discretized Problem}


\subsection{Refs}
[1]: https://en.wikipedia.org/wiki/Finite\_difference


\begin{table}[h]
	\caption{Wall-clock time (in seconds) and speed-up (in brackets) using multiple cores on Euler for solving the Poisson problem.}
	\centering
	
	\begin{tabular}{l|r||r|r|r|r}\hline\hline
		Problem & \multicolumn{1}{c||}{$N$} &  \multicolumn{4}{c}{Number of Euler cores} \\
		&       & \multicolumn{1}{c|}{1} & \multicolumn{1}{c|}{8} & \multicolumn{1}{c|}{16} & \multicolumn{1}{c}{32} \\
		\hline\hline
		{ Poisson} & $500^2$  &    \phantom{222222}        &    \phantom{222222}      & \phantom{222222}         &      \phantom{222222} \\
		{ Poisson} & $1000^2$ &            &          &          &       \\
		{ Poisson} & $2000^2$ &            &          &          &       \\
		{ Poisson} & $3000^2$ &            &          &          &       \\\hline \hline
	\end{tabular}
	
	\label{tab:PDEparallel1}
\end{table}


% -------------------------------------------------------------------------- %
% -------------------------------------------------------------------------- %
% --- Exercise 2 ----------------------------------------------------------- %
% -------------------------------------------------------------------------- %
% -------------------------------------------------------------------------- %

\section{Interactive Supercomputing using Jupyter Notebook  [10 points]}


% -------------------------------------------------------------------------- %
% -------------------------------------------------------------------------- %
% --- Exercise 3 ----------------------------------------------------------- %
% -------------------------------------------------------------------------- %
% -------------------------------------------------------------------------- %


\section{Jupyter Notebook - Parallel PDE-Constrained Optimization [40 points]}

\begin{table}[h]
  \caption{Wall-clock time (in seconds) and speed-up (in brackets) using multiple cores on Euler for solving the PDE-constrained optimization problem.}
	\centering
	
	\medskip
	
	%\footnotesize
	\begin{tabular}{l|r||r|r|r|r}\hline\hline
		Problem & \multicolumn{1}{c||}{$N$} &  \multicolumn{4}{c}{Number of Euler cores} \\
		&       & \multicolumn{1}{c|}{1} & \multicolumn{1}{c|}{8} & \multicolumn{1}{c|}{16} & \multicolumn{1}{c}{32} \\
		\hline\hline
		{ Inverse Poisson} & $500^2$  &    \phantom{222222}        &    \phantom{222222}      & \phantom{222222}         &      \phantom{222222} \\
		{ Inverse  Poisson} & $1000^2$ &            &          &          &       \\
		{ Inverse Poisson} & $2000^2$ &            &          &          &       \\
		{ Inverse Poisson} & $3000^2$ &            &          &          &       \\\hline \hline
	\end{tabular}
	
	\label{tab:PDEparallel}
\end{table}

\section{Task:  Quality of the Report [15 Points]}
Each project will have 100 points (out of  which 15 points will be given to the general quality of the written report).


\section*{Additional notes and submission details}
Submit the source code files (together with your used \texttt{Makefile}) in
an archive file (tar, zip, etc.), and summarize your results and the
observations for all exercises by writing an extended Latex report.
Use the Latex template provided on the webpage and upload the Latex summary
as a PDF to \href{https://moodle-app2.let.ethz.ch/course/view.php?id=14316}{Moodle}.

\begin{itemize}
	\item Your submission should be a gzipped tar archive, formatted like project\_number\_lastname\_firstname.zip or project\_number\_lastname\_firstname.tgz. 
	It should contain
	\begin{itemize}
		\item all the source codes of your solutions;
		\item your write-up with your name  project\_number\_lastname\_firstname.pdf.
	\end{itemize}
	\item Submit your .zip/.tgz through Moodle.
\end{itemize}

\end{document}
